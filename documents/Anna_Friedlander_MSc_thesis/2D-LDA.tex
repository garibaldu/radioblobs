\chapter[Dirichlet-multinomial score with LDA output]{The Dirichlet-multinomial score using LDA output}\label{C:2D-LDA}

This chapter presents an exploration into combining latent Dirichlet allocation (Chapter \ref{C:LDA}) with the Dirichlet-multinomial score (Chapters \ref{C:1D} and \ref{C:2D}) on astronomical images discretised by histogram binning (Chapter \ref{C:BIN}).

LDA and the DM-Score were combined by using the unnormalised source and background distributions\footnote{Where source and background distributions are mulitnomial distributions over bins.} derived by LDA as the $\boldsymbol{\alpha}^S$ and $\boldsymbol{\alpha}^B$ parameters in the DM-Score. These unnormalised topic distributions are the per-bin counts of pixels for each topic, as estimated by Gibbs sampling.

Because the Dirichlet distribution is the conjugate prior to the multinomial distribution\footnote{That is, if a data point $x$ comes from a multinomial distribution, and the prior for its parameter $p(\theta)$ comes from a Dirichlet distribution, then the posterior distribution of its parameter given the data point $p(\theta|x)$ also comes from a Dirichlet distribution.}, the unnormalised topics can be viewed as pseudocounts in $K$ bins, which can be treated as an $\boldsymbol{\alpha}$-vector from a Dirichlet distribution\footnote{This is how the topic distributions are treated in the Gibbs sampling process in LDA.} \cite{kotz2004continuous,ng2011dirichlet,steyvers2007probabilistic}.

\section{Methods}
Test data comprised five windows of $500 \times 500$ pixels from the two large astronomical images \cite{norris2006deep}, chosen randomly from the set of $25$  used for the evaluation of the DM-Score in Chapter \ref{C:2D}. Sources were labelled by the source-finding package BLOBCAT \cite{hales2012blobcat} with manual postprocessing by an astronomer. The windows are shown in Figure \ref{fig:real-data-2d-ldadm}.

\begin{figure}
\centering
\includegraphics[width=1.0\textwidth]{IMAGES/LDA-gt2.png}
\caption[Astronomical images used for evaluation]{\textbf{Astronomical images used for evaluation.} The central coordinates and a bounding box around each ground-truth source shown in green. The images are $500 \times 500$ windows from ELAIS (top left: ELAIS W1) and CDFS (top right: CDFS W1, bottom left: CDFS W2, bottom right: CDFS W3) \cite{norris2006deep}. Images are contrast adjusted with colour inverted to show sources.}
\label{fig:real-data-2d-ldadm}
\end{figure}

The images were discretised by the equal width histogram binning method described in Chapter \ref{C:BIN}. 

Latent Dirichlet allocation, as described in Chapter \ref{C:LDA}, was run to extract multinomial distributions over bins for source and background.

These distributions over bins were then used as $\boldsymbol{\alpha}^S$ and $\boldsymbol{\alpha}^B$ parameters in the DM-Score, and the overall topic proportions derived by LDA were used in place of the second term in Equation \ref{eq:score2}.

In contrast to the symmetric $\boldsymbol{\alpha}^S$-vector with $\alpha_i = 1 \; \forall i$ used in the DM-Score in Chapter \ref{C:2D}, the $\boldsymbol{\alpha}^S$-vector derived by LDA is not symmetric, but skewed towards brighter pixel intensities relative to the $\boldsymbol{\alpha}^B$-vector (while still having counts in the darker intensities). In the DM-Score in Chapter \ref{C:2D}, all multinomial distributions over $K$ bins are equally likely to be drawn from the Dirichlet distribution for source, while the $\boldsymbol{\alpha}^S$-vector derived by LDA is more likely to produce distributions with higher likelihood for brighter pixels. 

Similarly, the $\boldsymbol{\alpha}^B$-vector derived by LDA is skewed towards darker pixels, even more so than using the whole binned image as the $\boldsymbol{\alpha}^B$-vector, which includes source pixels as well as background pixels.

Gradient ascent, as described in Chapter \ref{C:2D}, was performed to find peaks in the score. 

Sources found at the end of each round of gradient ascent were removed from the data, but in contrast to the gradient ascent procedure for DM-Score in Chapter \ref{C:2D} the $\boldsymbol{\alpha}^B$-vector and bin borders were not recalculated. In Chapter \ref{C:2D} the entire image was used as a proxy for background. In this context, removing source pixels from the $\boldsymbol{\alpha}^B$ parameter should improve the background model. However if the background distribution derived from LDA is regarded as a good model, the $\boldsymbol{\alpha}^B$ parameter should not need to be recalculated.

The performance of the combined method was compared to the performance of the DM-Score using equal width bin borders\footnote{The equal width binning strategy is the strategy used for LDA in this chapter. Using the same binning strategy to compare the results for the DM-Score using LDA output with the DM-Score as described in Chapter \ref{C:2D} allows a fair comparison of the two methods.} using the procedure for real two dimensional data described in Chapter \ref{C:2D}.

Precision and recall were calculated as in Chapter \ref{C:2D}, with the same definitions for true positives, false positives and false negatives.

\section{Results}

\begin{figure}
\centering
\includegraphics[width=1.0\textwidth]{IMAGES/LDA-fnd2.png}
\caption[Results of the DM-Score combined with LDA]{\textbf{Found sources.} Sources found by the DM-Score with $\boldsymbol{\alpha}$ parameters set using source and background topics derived by LDA. The sources in the ground-truth catalogue in each image are shown in Figure \ref{fig:real-data-2d-ldadm}. An example of a low surface brightness source found by the DM-Score but not in the ground truth catalogue is circled in blue.}
\label{fig:results-all}
\end{figure}

Some preliminary results of this technique are presented in Table \ref{table:2d-lda-real}. In general, precision and recall are higher than the results for the DM-Score alone (that is, the DM-Score without LDA topic distributions as $\boldsymbol{\alpha}$-vectors), as well as higher than the results in Chapter \ref{C:2D}. Given the small sample size of five images, this can only be taken as a preliminary indication of success of the combined method. More investigation and comparison between the two methods, and of the DM-Score with different $\boldsymbol{\alpha}^S$ and $\boldsymbol{\alpha}^B$-vectors, is needed.

\begin{table}
\centering
\caption[Performance of DM-Score using LDA output]{Performance of DM-Score using LDA output}
\begin{tabular}{l| c c| c c}
\hline
Subwindow & \multicolumn{2}{c|}{DM-Score alone} & \multicolumn{2}{|c}{using LDA output} \\ 
          & Precision & Recall& Precision & Recall \\ \hline
ELAIS W1  & 0.71 & 0.59 & 0.87 & 0.90\\ %3051_5982
CDFS W1   & 0.75 & 0.91 & 0.77 & 0.93 \\ %4548_5598
CDFS W2   & 0.51 & 0.93 & 0.55 & 1.0 \\ %4135_3210
CDFS W3   & 0.66 & 0.71 & 0.70 & 0.76 \\ %3797_4862
CDFS W4   & 0.29 & 0.86 & 1.0  & 0.71 \\\hline %2218_4191 
Total (all windows) & 0.58 & 0.85 & 0.72 & 0.87 \\\hline
\end{tabular}
\label{table:2d-lda-real}
\end{table}

In particular, precision for the DM-Score with LDA output is much higher than the precision for the DM-Score alone. Note however that the ground truth catalogue included only sources at least $5 \sigma$ above rms noise. The combined method in this chapter therefore found \textit{no} sources below this threshold in window CDFS W4, which yielded precision of 1.0 on the ground truth catalogue.

For the other images, however, in which precision is lower, there are examples of sources identified that are missing from the ground truth catalogue (though in some cases noise was identified as sources). Figure \ref{fig:results-all} shows a low surface brightness radio galaxy tail, with intensity less than $3 \sigma$ above rms noise, identified by the methods in this chapter but missing from the ground truth catalogue.

\section{Discussion and future work}
The preliminary results in Table \ref{table:2d-lda-real} show good results for recall and precision, much higher than DM-Score alone. 

Like the DM-Score in Chapter \ref{C:2D}, while some areas of noise were falsely identified as sources, many so called ``false positives" --- identified by the combined method in this chapter but not in the ground truth catalogue --- are actual sources that are missing from the catalogue.

The ground truth catalogue contains only sources of intensity at least $5 \sigma$ above rms noise. The nature of pixel-intensity based thresholding algorithms such as BLOBCAT restricts the ability of the algorithms to find dim sources without also finding such a large number of noise regions that the results are unusable.

However, some of the most scientifically important objects in astronomy are dim, with intensities in the range of background noise \cite{norris2011emu}. Figure \ref{fig:results-all} shows an example of a low surface brightness object identified by the methods in this chapter but missing from the ground truth catalogue. This is an object that would likely to also be missed by LDA alone --- the implementation of LDA in Chapter \ref{C:LDA} had high rates of precision and recall for sources with high intensity pixels, but, as concluded in that chapter, was unlikely to identify low surface brightness sources.

In contrast to the low surface brightness sources identified, the combined method in this chapter found \textit{no} sources below the $5 \sigma$ threshold in window CDFS W4. This suggests that the values with which the $\boldsymbol{\alpha}$-vectors are set could be possibly manipulated to set thresholds in the source finding task.

Given the encouraging results of this preliminary investigation into the combination of LDA and the DM-Score, more experiments would be worthwhile pursuing.
