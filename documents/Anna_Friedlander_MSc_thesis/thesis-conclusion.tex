\chapter{Conclusions}\label{C:con}

\section{Summary}
This thesis explored Bayesian methods for source detection that use Dirichlet or multinomial models for pixel intensity distributions in discretised radio astronomy images. 
\begin{itemize}
\item A novel image discretisation method that incorporates uncertainty about how the image should be discretised was developed.
\item Latent Dirichlet allocation --- a method originally developed for inferring latent topics in document collections --- was used to estimate source and background distributions in radio astronomy images. 
\item A new Dirichlet-multinomial score, indicating how well a region conforms to a well-specified model of background versus a loosely-specified model of foreground, was derived.
\item Latent Dirichlet allocation and the Dirichlet-multinomial score were combined for source detection in astronomical images. 
\end{itemize}

The methods developed in this thesis perform source detection well in comparison to two widely-used source detection packages, and importantly, find scientifically important dim sources not well found by other other algorithms, while returning only a moderate number of false positives in regions of noise.

\section{Future work}
This thesis included initial explorations of a number of techniques. The results of these early experiments reveal some of the shortcomings of the techniques and suggest a number of avenues for future work. These are summarised below.

\subsection{Image discretisation}

In this thesis, only simple width and frequency based histogram binning methods (with and without the Dirichlet softening of bin borders) were considered. These approaches may be problematic when the underlying distribution of pixel intensities is not uniform \cite{liu2002discretization}, as is the case with radio astronomy images (Figure \ref{fig:histo}). These strategies can consign pixels that would otherwise be discriminatory into histogram bins such that their informativity is destroyed. For example, low brightness source pixels may become lost among background pixels. 

Bright outliers can also skew the binned representation of the data\footnote{In the case of equal width bins with iterative source removal and rebinning, as in the DM-Score, this may actually be an advantage. Isolating the brightest pixels highlights them, and removing the sources that contain these pixels allow the next brightest pixels to be revealed and so on until (ideally) all sources have been removed and only background pixels remain. This may explain the success of the equal width binning schemes in comparison to the equal occupancy schemes on astronomical images in Chapter \ref{C:2D}.}.

It would be worth exploring other, potentially more principled, approaches to image discretisation. A binning scheme that preserves the underlying distribution of the data would be worth investigating. Such discretisation methods could be evaluated on the basis of whether using them improved source detection results in techniques such as LDA and the DM-Score.

With regards to the Dirichlet bin-border softening developed in this thesis, more exploration of the $\boldsymbol{\alpha}$-vector used to create the borders should be undertaken. The symmetric $\boldsymbol{\alpha}$-vector with $\alpha_i = 10$, which produces fairly uniform distributions (\cite{frigyik2010introduction}; Figure \ref{fig:simplex}) was used in all cases of Dirichlet softening in this thesis. Symmetric vectors with different values could be explored, as could asymmetric vectors with values that cause some bin borders to be more variable than others.

\subsection{Latent Dirichlet allocation}

Though there were some good early results of LDA, with LDA performing similarly to \emph{Duchamp} in detecting sources for which the total flux (intensity) is less than 1.63 mJy, the implementation of LDA in this thesis is insufficient to find the scientifically important sources with intensities close to background. There are a number of avenues for exploration that could potentially address this issue.

In many cases, final background and source topics derived by LDA were simple thresholds in bins, with the background distribution over bins placing almost all weight on bins from $0$ to some bin $i$, and the source distribution placing almost all weight from bin $i$ to bin $K$. In this way, the implementation of LDA in this thesis is no better than pixel-intensity based thresholding algorithms that restrict search to those sources that contain pixels above a particular threshold. 

This thresholding behaviour may be due to the binning scheme used (equal width binning), and so an investigation into a better binning scheme may help. The implementation of LDA in this thesis does not allow for the use of Dirichlet bin borders, as the Gibbs sampling algorithm on the conditional distribution of topic allocations does not work with partial counts. The use of Dirichlet bin borders in LDA should be investigated.

Thresholding may also be due to the Dirichlet priors over topic distributions, which have $\alpha$ values $<1$. Such values produce sparse distributions (\cite{frigyik2010introduction}; Figure \ref{fig:simplex}) over bins; using an $\boldsymbol{\alpha}$-vector that encourages more even distributions over bins may help address the thresholding observed. 

Pixels are treated as fundamental units (words) in the implementation of LDA in this thesis. As the resolution element of a radio telescope is generally several pixels, individual pixels may be better thought of as having sub-word size (perhaps ``syllables"). The implementation of LDA may be adjusted accordingly.

The use of the final topic distributions derived by LDA --- segmentation by assigning each pixel a hard topic label and source detection by flood-filling on the segmented image --- is crude. A more nuanced approach would eliminate this hard assignment and take a more probabilistic approach to region labelling. For example, given the multinomial models for background and source, gradient ascent could be performed to find regions that have high likelihood under a particular model.

\subsection{The Dirichlet-multinomial score}

Again, an investigation into the optimal binning scheme is needed, as no clear conclusions can be drawn on this point given the results of the DM-Score in this thesis. 

The $\boldsymbol{\alpha}$-vectors in the DM-Score could similarly be investigated. In this thesis, the $\boldsymbol{\alpha}^S$-vector was set to the symmetric $\boldsymbol{\alpha}$-vector with $\alpha_i = 1$, in order to give equal likelihood to all multinomial distributions over $K$ for source regions (defining source regions as any ``non-background" region). It would be useful to explore ways of setting the $\boldsymbol{\alpha}^S$-vector such that particular distributions are favoured over others. 

With regards to the background compensation correction to the score, initial investigation indicated that this correction improves results for one dimensional but not two dimensional data. The reasons for this are not clear and further investigation is warranted.

\subsubsection{Procedural issues}

The gradient ascent process used in this thesis was too slow to be practical (or even possible) on full-sized large images. If the DM-Score is to be usable on the large amounts of data produced by next generation telescopes, speed must be addressed. The use of a Gaussian distribution for test regions meant that regions extend potentially across the whole image, slowing down gradient ascent. One aspect that could be improved to speed up the gradient ascent process therefore is the use of a test region with finite support. 

The iterative removal of sources found by the DM-Score sometimes left artefacts around the removed source. This can cause multiple ``found sources" where in fact only one source lies (Figure \ref{fig:false-positives}). Addressing this issue may reduce the number of false positives the DM-Score yields.

\subsubsection{Extensions}

With regards to future extensions to the score, the second term in the DM-Score (Equation \ref{eq:score2}) weights the likelihood of a region being source or background by the relative proportion of each in the image. This term is set heuristically in this thesis, to reflect the prior belief that astronomical images are dominated by background pixels. However, a Dirichlet hyperparameter could be placed over the source and background mixing proportions in an image, and inferred, rather than set heuristically.

\subsection{The Dirichlet-multinomial score using LDA output}

Early experiments into the combination of LDA with the DM-Score suggest better performance than either algorithm alone. More evaluation to confirm this result would be worthwhile. 

On the other hand, like LDA, the combination of LDA with the DM-Score seems to miss a number of low intensity sources. As with LDA, an investigation into an optimal binning scheme may address this.

Though sources were iteratively removed with this version of the DM-Score, there was no recalculation of the $\boldsymbol{\alpha}$-vectors or of the bin borders. Doing so may improve performance on dim sources. However unlike the DM-Score implementation in Chapter \ref{C:2D} where the initial $\boldsymbol{\alpha^B}$-vector and bin borders were calculated using the whole image as a proxy for background, there is no readily apparent way of performing this iterative recalculation. One possible approach would be to simply reduce the counts in the $\boldsymbol{\alpha^S}$ and $\boldsymbol{\alpha^B}$-vector bins corresponding to the bins of the pixels of the removed source. Further exploration into rebinning schemes would be worthwhile.





