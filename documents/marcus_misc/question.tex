\documentclass[12pt]{article}
% This is Marcus's style file, for little research notes and so on.  I
% keep this in $HOME/texmf/tex/latex, because kpsewhich
% -var-value=TEXMFHOME tells me that ($HOME/texmf) is where latex
% looks, at least under Ubuntu 12.10

\renewcommand{\familydefault}{\sfdefault} % use sans serif fonts
\usepackage{graphicx}
\usepackage[usenames,dvipsnames]{color}% so you can {\color{red} anything}, etc
\usepackage[fleqn]{amsmath}
\usepackage{amsmath}
\usepackage{amssymb,amsfonts}
\setlength{\textwidth}{16cm}
\setlength{\textheight}{24cm}
\hoffset=-2cm
\voffset=-2cm
\raggedbottom
\setlength{\parindent}{0in}
\setlength{\parskip}{0.14in plus 0.07in minus 0.07in}

\setcounter{tocdepth}{2} % determines the depth to which tableofcontents goes

%--------------------------------
\newcommand{\Line}[0]{\rule{0cm}{0cm}\\\hrule\rule{0cm}{0cm}}
\newcommand{\independent}{\protect\mathpalette{\protect\independenT}{\perp}}
\def\independenT#1#2{\mathrel{\rlap{$#1#2$}\mkern2mu{#1#2}}}
\newcommand{\dependent}{\not\independent}
\newcommand{\data}{\mathcal{D}}
\newcommand{\scriptL}{\mathcal{L}}
\newcommand{\mat}[1]{\ensuremath{\mathbf{#1}}} % vectors/matrices, x,A,B etc.
\newcommand{\gmat}[1]{\mbox{\boldmath${#1}$}}  % vectors/matrices Greek symbols
\newcommand{\swrt}[2]{\frac{\partial #1}{\partial #2}} % partial derivative of single symbol
\newcommand{\bwrt}[1]{\frac{\partial}{\partial #1}} % partial derivative of something big
\newcommand{\bA}{\mat{A}}
\newcommand{\bh}{\mat{h}}
\newcommand{\bw}{\mat{w}}
\newcommand{\bv}{\mat{v}}
\newcommand{\bx}{\mat{x}}
\newcommand{\by}{\mat{y}}
\newcommand{\bz}{\mat{z}}
\newcommand{\bp}{\mat{p}}
\newcommand{\br}{\mat{r}}
\newcommand{\bs}{\mat{s}}
\newcommand{\bt}{\mat{t}}
\newcommand{\bu}{\mat{u}}
\newcommand{\bm}{\mat{m}}
\newcommand{\bn}{\mat{n}}
\newcommand{\bc}{\mat{c}}
\newcommand{\bC}{\mat{C}}
\newcommand{\bH}{\mat{H}}
\newcommand{\bM}{\mat{M}}
\newcommand{\bT}{\mat{T}}
\newcommand{\bW}{\mat{W}}
\newcommand{\bX}{\mat{X}}
\newcommand{\bY}{\mat{Y}}
\newcommand{\bZ}{\mat{Z}}
\newcommand{\balpha}{\boldsymbol\alpha}
\newcommand{\bSigma}{\boldsymbol\Sigma}
\newcommand{\btheta}{\boldsymbol\theta}
\newcommand{\wo}{\mat{\backslash}}
\newcommand{\bCinv}{\mat{C}^{-1}}
\newcommand{\bCinvPrime}{\mat{C^\prime}^{-1}}
\newcommand{\bkNN}{\mat{k}}
\newcommand{\T}{\ensuremath{\textit{T}}}
\newcommand{\Dir}{\ensuremath{\text{Dir}}}

\newcommand{\Bel}[0]{\text{Bel}}
\newcommand{\on}[0]{\text{on}}
\newcommand{\off}[0]{\text{off}}
\newcommand{\ith}[0]{\text{$i^\text{th}\;$}}
\newcommand{\jth}[0]{\text{$j^\text{th}\;$}}
\newcommand{\kth}[0]{\text{$k^\text{th}\;$}}
\newcommand{\wth}[0]{\text{$w^\text{th}\;$}}
\newcommand{\dth}[0]{\text{$d^\text{th}\;$}}
\newcommand{\nth}[0]{\text{$n^\text{th}\;$}}
\newcommand{\tth}[0]{\text{$t^\text{th}\;$}}

\newcommand{\half}[0]{\text{\tiny$\frac{1}{2}$}}
\newcommand{\onethird}[0]{\text{\tiny$\frac{1}{3}$}}
\newcommand{\twothirds}[0]{\text{\tiny$\frac{2}{3}$}}

\newcommand{\real}{\mathbb{R}}
\newcommand{\reals}{\mathbb{R}}
\newcommand{\Expectation}{\mathbb{E}}
\newcommand{\expectation}{\mathbb{E}}
\newcommand{\obs}{\mathrm{obs}}
\newcommand{\tinyhalf}{\mbox{\tiny $\frac{1}{2}$}}

\newcommand{\tick}{\ding{51}}
\newcommand{\btick}{\ding{52}}
\newcommand{\cross}{\ding{55}}
\newcommand{\bcross}{\ding{56}}
\newcommand{\given}{\,|\,}

\renewcommand{\vec}[1]{\mathbf{#1}}
%\newcommand{\balpha}{\mbox{\boldmath $\alpha$}}
%\newcommand{\btheta}{\mbox{\boldmath $\theta$}}
%\newcommand{\bphi}{\mbox{\boldmath $\phi$}}
%\newcommand{\bn}{\mbox{\boldmath $n$}}
\newcommand{\DCM}{\mbox{DCM}}

  
% I keep MarcusStyle.sty in $HOME/texmf/tex/latex, because 
% kpsewhich -var-value=TEXMFHOME says $HOME/texmf is where latex looks.

%\setlength{\parskip}{2mm plus 1mm minus 1mm}
\renewcommand{\arraystretch}{1.5}
\renewcommand*{\familydefault}{\sfdefault}

\title{relating Bayes Factor to a ``full'' generative model for images}
\author{marcus}
\date{}
\begin{document}
\maketitle


\section{a generative model of astro images}

\subsection{basics of DCM distribution}
(following Wikipedia...) ``DCM'' stands for the Dirichlet compound
multinomial distribution over an $I$-dimensional vector of counts
$\bn$:

\begin{align}
\DCM(\bn|\balpha) &= \frac{\Gamma(A)}{\Gamma(N+A)} \prod_i^I \frac{\Gamma(n_i+\alpha_i)}{\Gamma(\alpha_i)}  
%\intertext{where}
\;\;\;\;\;\;\;\;\; \text{with} \;A = \sum_i^I \alpha_i, \;\;\; N = \sum_i^I n_i
\notag \intertext{Taking logs,}
\log \DCM(\bn|\balpha) &= \log \Gamma(A) - \log \Gamma(N+A) + \sum_i \log \Gamma(n_i+\alpha_i) - \log \Gamma(\alpha_i) \label{eq:logmultdir}
\end{align}

In our case, the counts $\bn$ are determined by the choice of region,
parameterised by $\theta$, and so $\bn = \bn(\theta)$. For a single
elliptical source, $\theta = (x,y,w_x,w_y,\phi)$ (ie. position,
widths, and rotation).

Denoting the derivative of the log gamma function
$\frac{\partial}{\partial n}\log \Gamma(n)$ by $\psi(n)$, the gradient
of $\log \DCM$ w.r.t. $\theta$ is
\begin{align}
\frac{\partial}{\partial \theta} \log \DCM(\bn(\theta) |\balpha) 
&= \sum_i W_i \, \frac{\partial n_i(\theta)}{\partial \theta} 
%\sum_i \bigg[  \psi(n_i+\alpha_i) - \psi(N+A) \bigg] \frac{\partial n_i}{\partial \theta}
\label{eq:gradientLogDCM}
\intertext{where $W_i = \psi(n_i+\alpha_i) - \psi(N+A)$.}
\end{align}


\subsection{our use of Bayes factor in source finding} 
FreanFriedlanderJohnsonHollitt2013 ({\sc FreFri}) used the two-class model in
which $\balpha^0$ has large numbers in it and corresponds to the
background, and $\balpha^1 = (1,1,\ldots,1)$ representing our
ignorance regarding the source distribution. We used the ratio of the
associated posterior probs (a.k.a. the ``Bayes factor'') for the pixel
values in a region as a ``score'' for the sourciness of that
region. The source parameters $\btheta$ could be thought of as
parameters specifying the border between source and background if you
like. We just have a strong prior we're building in that this border
tends to be elliptical.  Our procedure is: move region parameters
$\btheta$ to increase Bayes Factor\footnote{Actually the ratio of
  \emph{posterior} probabilities rather than just the likelihoods,
  which leads to an additional additive constant.} $\log
\frac{\DCM(\vec{n}^1 \given \balpha^1)}{\DCM(\vec{n}^1 \given
  \balpha^0)}$.  Thus optimizing the score in the space of region
parameters amounts to source finding.


\subsection{generative model}
Is our current scheme, in which we search for regions that have high
Bayes Factor values, equivalent or close to optimizing a parameterised
generative model of the \emph{image as a whole}? We haven't thought about it
in this way before, but writing down such a model would help us to
generalise the scheme to more than two classes.

A procedure for generating an image containing a \emph{single source}
`is as follows, assuming we're binning the pixel intensities into $I$ bins:
\begin{enumerate}
\item make up parameters $\btheta$ describing a region of the image.
\item note that $\btheta$ determines the numbers of pixels inside ($N^1$)  and  outside ($N^0$) the region.
\item set up $\balpha^1 \in \mathbb{R}^I$, with small numbers,
  and $\balpha^0 \in \mathbb{R}^I$, with big numbers, especially at the bins corresponding to low amplitudes.
\item the bin counts are DCM distributed, but subject to constraints on $\sum_i n_i$ determined by the choice of $\btheta$.
  \begin{itemize}
    \item    $\vec{n}^1 \sim \DCM(\vec{n} \given \balpha^1)$
    \item $\vec{n}^0 \sim \DCM(\vec{n} \given \balpha^0)$
  \end{itemize}
\end{enumerate}


\begin{figure}
\includegraphics[scale=1.0]{./pics/FreFriPGM}
\caption{Here's what I think the {\sc FreFri} PGM looks like. $\beta$ is just some prior over $\theta$ (eg. we have an inkling of what typical sizes might be, and limits on how elliptical, whateva). Of the three latent variables, the two grey nodes are integrated out analytically through the magic of the DCM distribution. And we {\it optimize} the orange one to find a specific region. The things I'd like to think about next are (i) how to represent an image with many many many sources, and (ii) how we might consider the $\alpha$ ``constants'' as learnable parameters instead.
\label{fig:FreFriPGM}
}
\end{figure}

So this generative model of an image takes parameters $\theta,
\balpha^0, \balpha^1$, and results in just one ``data point''
consisting of the two vectors of counts $\vec{n}^0,\vec{n}^1$, which are sampled independently of one another.  The
likelihood and its logarithm are therefore
\begin{align*}
L =& \DCM(\vec{n}^0 \mid \balpha^0) \; \cdot \; \DCM(\vec{n}^1 \mid \balpha^1) 
\\ \\
\log L =& \log \DCM(\vec{n}^0 \mid \balpha^0) \;\; + \;\; \log \DCM(\vec{n}^1 \mid \balpha^1) 
%\\ \\ =& \log \Gamma(A^0) +\log \Gamma(A^1) - \log \Gamma(N^0+A^0)  - \log \Gamma(N^1+A^1) + \\ & \sum_i \log \Gamma(n^0_i + \alpha_i^0) - \log \Gamma(\alpha_i^0) + \sum_i \log \Gamma(n^1_i + \alpha_i^1) - \log \Gamma(\alpha_i^1) 
\end{align*}

This doesn't seem much like our Bayes factor though, since it
\emph{adds} instead of subtracts the two $\log \DCM$ terms! The terms
are slightly different too.  Here they are, for direct comparison:

\begin{tabular}{|l|l|}
\hline
log DMR (Bayes Factor): & 
\parbox{.7\textwidth}{
\begin{align*}
&\log \DCM(\vec{n}^1 \given \balpha^1) \;\;-\;\; \log \DCM(\vec{n}^1 \given \balpha^0)
\end{align*}
} \\
\hline
Log L: & 
\parbox{.7\textwidth}{
\begin{align*}
& \log \DCM(\vec{n}^1 \mid \balpha^1) \;\; + \;\; \log \DCM(\vec{n}^0 \mid \balpha^0)
\end{align*}
} \\
\hline
\end{tabular}

Note that with $\log L$ it seems that we're modelling the whole image
in that $\bn^0$ is involved, but with the Bayes factor we are only
considering the current region. But in fact this is not so, and {\it
  the two are equivalent} up to an additive constant (so far as
$\btheta$ is concerned).

Consider the log likelihood of the entire image under just the
``background'' model (or alternatively, consider the log likelihood in the case that we were to set $\balpha^1$ equal to $\balpha^0$):
\begin{align*}
\log \DCM(\bn \mid \balpha^0) 
&= \log \DCM(\bn^0(\theta) + \bn^1(\theta) \mid \balpha^0)  \\
&= \log \DCM(\bn^0(\theta) \mid \balpha^0) \; + \; \log \DCM(\bn^1(\theta) \mid \balpha^0) 
\end{align*}
This \emph{has to} be unaffected by our choice of region $\btheta$,
ie. its derivative w.r.t. $\theta$ must be zero everywhere.  Adding
this constant to the original log likelihood immediately yeilds the
``Bayes factor'' score we've been using!

Put another way, the gradient w.r.t. any dimension of $\theta$ for the
log likelihood under the above generative model is the same as the
gradient of the ratio we used.  For the record, the gradient of our
Bayes factor (and thus the gradient of the log likelihood too) is...
\begin{align}
\frac{\partial}{\partial\theta}\text{DMR}(\theta) 
&= \sum_i \big[ \psi(n_i + \alpha^S_i) - \psi(n_i + \alpha^B_i) \big] \frac{\partial n^1_i}{\partial\theta} \notag\\
& - \;\; \big[\psi(N^1+A^1) - \psi(N^1+A^0)]\sum_i \frac{\partial n^1_i}{\partial\theta}
\end{align}


Describing things in terms of (the likelihood of) a generative model
for the image as a whole is going to be more useful in some contexts,
and especially for thinking about how to get beyond one source, and
two classes.

\section{But that's actually not correct!!!}

Consider a vector whose elements are categorical variables $y \in \mathcal{Y} $ where $\mathcal{Y} = \{Y_1,\ldots,Y_K \}$
\[
\by = (y_1,y_2, \ldots, y_N) \]
We will use $\mathbb{Y}$ to denote the same as a set (ie. without the ordering imposed by the vector). The number of times each category occurs (its "count") forms a vector
\[
\bn = n_1,\ldots,n_K  
\] 
and we'll denote the {\it total} count $\sum_{k=1}^K n_k$ by $N$.

Now consider splitting the ``image'' $\by$ into two parts, e.g. are some pixel mid-way along its length:
\begin{align*}
\by =& (\by^A, \bz^B)
\intertext{To connect this to radioblobs, think of $\by^A$ as the values inside a putative blob region and $\by^B$ as the rest (the "background"). 
At any rate, I initially thought one could just write
}
P(\by) =& P(\by^A , \by^B)\\
=& P(\by^A) \; P(\by^B)
\end{align*}
and I think I've assumed that in the foregoing.

Eg: The vector \[\mathtt{000000011111111111111}\] has exactly the same
probability using a fair coin as \[\mathtt{10110111101111011100}\]
but what about the probability of 
tossing \[\mathtt{0000000} \;\textit{ followed by  } \;\mathtt{11111111111111}\]?

IF $\by$ elements are all drawn from the same Bernoulli distribution
with parameter $p$ of generating a ``1'', then we have the binomial distribution for the
counts:
\begin{align*}
\Pr(n | N,p) &= {N \choose n} \;\; p^n (1-p)^{N-n}
\;\;\;\;\;\; \text{where} \;\;{N \choose n} = \frac{N!}{n! (N-n)!} 
\intertext{The probs for the count of "1" within the sub-set $A$ is}
\Pr(n^A | N^A,p) &= \frac{N^A!}{n^A! (N^A-n^A)!} p^{n^A} (1-p)^{N^A-n^A}
\intertext{and similarly for the $B$ sub-set, and these are independent so we have}
\Pr(n^A, n^B| N^A,N^B,p) &= \frac{N^A!}{n^A! (N^A-n^A)!} \;\; \frac{N^B!}{n^B! (N^B-n^B)!} \;\; p^{n^A} (1-p)^{N^A-n^A} \;\;p^{n^B} (1-p)^{N^B-n^B} \\
&= \frac{N^A! \, N^B!}{n^A! n^B! \, (N^A-n^A)! (N^B-n^B)!}  \;\; p^{n} (1-p)^{N-n} \\
&= {N^A \choose n^A} \;\; {N^B \choose n^B} \;\; 
 \;\; p^{n} (1-p)^{N-n}
\end{align*}
So the terms involving $p$ match those of $\Pr(n | N,p)$, but not the
``n choose k'' factor at the front that captures the number of
equivalent permutations.


\section{Fun with Dirichlet}
Dirichlet Compound Multinomial (DCM) distribution has parameters
$\balpha = \alpha_1,..,\alpha_K$ with $\alpha_k > 0$ and we denote $ A
= \sum_k^K \alpha_k$.

Under the DCM distribution, we have that
\begin{align}
P(\mathbb{X} |\balpha) &= \frac{\Gamma(A)}{\Gamma(N+A)} \prod_k \frac{\Gamma(n_k+\alpha_k)}{\Gamma(\alpha_k)}  \label{eq:DCMvalues}
\end{align}

The gamma function just generalises the factorial to the reals: at
positive integer values they match except that the argument for gamma is
larger by one: $\Gamma(z) = (z-1)!$.

The probability of {\it counts} is almost the same, but with a multiplier:
\begin{align}
P(\bn |\alpha) &= \frac{N!}{\prod_k (n_k!)} \;
 \frac{\Gamma(A)}{\Gamma(N+A)} \;\prod_k \frac{\Gamma(n_k+\alpha_k)}{\Gamma(\alpha_k)}  \label{eq:DCMcounts}
\end{align}
I think I'm right in saying that the multiplier is the ``multinomial coefficient''
\[
\frac{N!}{\prod_k (n_k!)} \;
\;\;\; = \;\;\; 
{n \choose n_1, n_2,\ldots,n_K}
\]
which is just a generalisation of the more familiar binomial coefficient ${n \choose k}$.

\subsection{a more succinct notation?}

The multinomial coefficient can be rewritten in terms of gamma functions:
\[
\frac{N!}{\prod_k (n_k!)} \;
\;\;\; = \;\;\; 
\frac{\Gamma(N+1)}{\prod_k \Gamma(n_k+1)}
\]
which has the same form as the other terms now.

Perhaps we can simplify things then. Let's define a function
\[
 \Omega(\bc) \;\;= \;\;\frac{\Gamma(C)}{\prod_k \Gamma(c_k)} \;\;=\;\; \frac{(C-1)!}{\prod_k (c_k-1)! }
\]
where $C=\sum_k c_k$.
Note the effect of incrementing one of the counts, say the one indexed $k^\star$:
\begin{align*}
\Omega(\bc) &\longrightarrow \Omega(\bc) \times \frac{C}{c_{k^\star}}
\intertext{whereas {\em de}crementing would change it to}
&\longrightarrow  \Omega(\bc) \times \frac{c_{k^\star}-1}{C-1}
\end{align*}


I can now write the DCM probability in equation \ref{eq:DCMcounts} using $\Omega$. Terms that come from the multinomial coefficient are shown in {\color{blue}blue}.
\begin{align}
P(\bn |\balpha) &= 
\frac{{\color{blue}\Omega(\bn+1)} \;\; \Omega(\balpha)}{ \Omega(\bn + \balpha)}
\label{eq:DCMcounts_inOmega}
\end{align}

Incrementing the $k^{\star}$-th count changes this to:
\begin{align}
P(\bn \mid \balpha) &\longrightarrow  
P(\bn \mid \balpha)\;\; \times \;\;{\color{blue}\bigg( \frac{N+1}{n_{k^\star}+1} 
\bigg)}
\;\,\bigg(\frac{n_{k^\star}+\alpha_{k^\star}}{N+A}
\bigg)\label{eq:increment}
\intertext{whereas {\em de}crementing would change it to}
&\longrightarrow  
P(\bn \mid \balpha)\;\; \times \;\;{\color{blue}\bigg( \frac{n_{k^\star}}{N} 
\bigg)}
\;\,\bigg(\frac{N+A-1}{n_{k^\star}+\alpha_{k^\star}-1} 
\bigg)\label{eq:decrement}
\end{align}


\subsection{moving the boundary}
Consider the ``split'' image again, with the left-hand side denoted
$\by$ and the right by $\bz$.

The effect of ``moving the boundary'' to the right by one pixel is to
take a single value that was in $\bz$ and move it into $\by$
instead. For generality, I'm going to distinguish between the
$\balpha$ used in generating $\by$ and the one used for $\bz$ too.

Using equations \ref{eq:increment} and \ref{eq:decrement}, the log likelihood of $\by$ changes by
\begin{align*}
\log P(\bn_\Delta^y \mid \balpha^y) -\log P(\bn^y \mid \balpha^y) 
&=
{\color{blue}\log(N^y+1) - \log(n^y_{k^\star}+1)}
+ \log(n^y_{k^\star}+\alpha^y_{k^\star}) - \log(N^y+A^y)
\intertext{whereas that of $\bz$ changes by}
\log P(\bn_\Delta^z \mid \balpha^z) - \log P(\bn^z \mid \balpha^z) 
&=
{\color{blue}\log(n^z_{k^\star}) - \log(N^z)} + \log(N^z+A^z-1) - \log(n^z_{k^\star}+\alpha^z_{k^\star}-1)
\end{align*}

SO IF (big if...) $P(\by,\bz \mid \text{model}) = P(\by \mid
\balpha^y) \, P(\bz \mid \balpha^z) $, this means the overall log
likelihood changes by
\begin{align*}
\Delta \log P &= \log P(\bn_\Delta^y,\bn_\Delta^z) - \log P(\bn^y,\bn^z) \\
&= 
{\color{OliveGreen}  
\log \bigg(\frac{n^y_{k^\star}+\alpha^y_{k^\star}}{N^y+A^y} \bigg) 
}
\; - \;
{\color{blue}  
  \log \bigg(\; 
  \frac{n^y_{k^\star}+1}{N^y+1}
  \frac{N^z} {n^z_{k^\star}}
\bigg)
  } 
\; - \; 
\log \bigg(\frac{n^z_{k^\star}+\alpha^z_{k^\star}-1}{N^z+A^z-1}
\bigg)
\end{align*}
where the $n,N$ are the {\it old} counts.


\subsection{Bayes factor}
Alternatively...

Here's the Bayes factor comparing the likelihood of $\by$ under the two possible $\balpha$ priors:
\begin{align*}
\frac{P(\bn^y \mid \balpha^y)}{P(\bn^y \mid \balpha^z)} \;
 &= \;
\frac{{\color{blue}\Omega(\bn^y+1)} \;\; \Omega(\balpha^y)}{ \Omega(\bn^y + \balpha^y)} \;\;\;
\frac{ \Omega(\bn^y + \balpha^z)}{{\color{blue}\Omega(\bn^y+1)} \;\; \Omega(\balpha^z)} 
\intertext{Note that the effect of the multinomial coefficient {\it cancels out}, leaving}
&= \;\frac{\Omega(\balpha^y)}{\Omega(\balpha^z)} 
\;\;\;
\frac{ \Omega(\bn^y + \balpha^z)}{ \Omega(\bn^y + \balpha^y)} 
\end{align*}

We (if this is correct) have been calculating the log of this quantity.


So let's consider the effect of moving the boundary on the Bayes factor:
\begin{align*}
\frac{P(\bn_\Delta^y \mid \balpha^y)}{P(\bn_\Delta^y \mid \balpha^z)}
\; &= \; \frac{\Omega(\balpha^y)}{\Omega(\balpha^z)} 
\;\;\;
\frac{ 
 \Omega(\bn^y + \balpha^z) \;\times\; \frac{N^y+A^z}{n^y_{k^\star}+\alpha^z_{k^\star}}
}{
 \Omega(\bn^y + \balpha^y) \;\times\; \frac{N^y+A^y}{n^y_{k^\star}+\alpha^y_{k^\star}}
} 
\intertext{where $\bn_\Delta$ denotes the {\it new} counts. So...}
\text{new BF} \;&= \;\text{old BF} \; \times \; 
\bigg(
 \frac{N^y+A^z}{n^y_{k^\star}+\alpha^z_{k^\star}}
\bigg)
\; \bigg(\frac{n^y_{k^\star}+\alpha^y_{k^\star}}{N^y+A^y}
\bigg)
\intertext{so in log space, here's how the Bayes Factor changes when a single pixel of class $X_{k^\star}$ gets moved from $\bz$ into $\by$:}
\Delta \text{BF} \;&= 
{\color{OliveGreen}  
\log\bigg(\frac{n^y_{k^\star}+\alpha^y_{k^\star}}{N^y+A^y}
 \bigg)
}
\;\;-\;\;
\log\bigg( \frac{n^y_{k^\star}+\alpha^z_{k^\star}}{N^y+A^z}
\bigg) 
\end{align*}
where the $n,N$ are the {\it old} counts.

{\color{red} RED THOUGHT: how about we define a source region as simply some set of
connected pixels, and individually ask all the pixels at the
borderline (both out and in) whether they want to join / leave?
What would happen?...}



\end{document}
