\documentclass[12pt]{article}
% This is Marcus's style file, for little research notes and so on.  I
% keep this in $HOME/texmf/tex/latex, because kpsewhich
% -var-value=TEXMFHOME tells me that ($HOME/texmf) is where latex
% looks, at least under Ubuntu 12.10

\renewcommand{\familydefault}{\sfdefault} % use sans serif fonts
\usepackage{graphicx}
\usepackage[usenames,dvipsnames]{color}% so you can {\color{red} anything}, etc
\usepackage[fleqn]{amsmath}
\usepackage{amsmath}
\usepackage{amssymb,amsfonts}
\setlength{\textwidth}{16cm}
\setlength{\textheight}{24cm}
\hoffset=-2cm
\voffset=-2cm
\raggedbottom
\setlength{\parindent}{0in}
\setlength{\parskip}{0.14in plus 0.07in minus 0.07in}

\setcounter{tocdepth}{2} % determines the depth to which tableofcontents goes

%--------------------------------
\newcommand{\Line}[0]{\rule{0cm}{0cm}\\\hrule\rule{0cm}{0cm}}
\newcommand{\independent}{\protect\mathpalette{\protect\independenT}{\perp}}
\def\independenT#1#2{\mathrel{\rlap{$#1#2$}\mkern2mu{#1#2}}}
\newcommand{\dependent}{\not\independent}
\newcommand{\data}{\mathcal{D}}
\newcommand{\scriptL}{\mathcal{L}}
\newcommand{\mat}[1]{\ensuremath{\mathbf{#1}}} % vectors/matrices, x,A,B etc.
\newcommand{\gmat}[1]{\mbox{\boldmath${#1}$}}  % vectors/matrices Greek symbols
\newcommand{\swrt}[2]{\frac{\partial #1}{\partial #2}} % partial derivative of single symbol
\newcommand{\bwrt}[1]{\frac{\partial}{\partial #1}} % partial derivative of something big
\newcommand{\bA}{\mat{A}}
\newcommand{\bh}{\mat{h}}
\newcommand{\bw}{\mat{w}}
\newcommand{\bv}{\mat{v}}
\newcommand{\bx}{\mat{x}}
\newcommand{\by}{\mat{y}}
\newcommand{\bz}{\mat{z}}
\newcommand{\bp}{\mat{p}}
\newcommand{\br}{\mat{r}}
\newcommand{\bs}{\mat{s}}
\newcommand{\bt}{\mat{t}}
\newcommand{\bu}{\mat{u}}
\newcommand{\bm}{\mat{m}}
\newcommand{\bn}{\mat{n}}
\newcommand{\bc}{\mat{c}}
\newcommand{\bC}{\mat{C}}
\newcommand{\bH}{\mat{H}}
\newcommand{\bM}{\mat{M}}
\newcommand{\bT}{\mat{T}}
\newcommand{\bW}{\mat{W}}
\newcommand{\bX}{\mat{X}}
\newcommand{\bY}{\mat{Y}}
\newcommand{\bZ}{\mat{Z}}
\newcommand{\balpha}{\boldsymbol\alpha}
\newcommand{\bSigma}{\boldsymbol\Sigma}
\newcommand{\btheta}{\boldsymbol\theta}
\newcommand{\wo}{\mat{\backslash}}
\newcommand{\bCinv}{\mat{C}^{-1}}
\newcommand{\bCinvPrime}{\mat{C^\prime}^{-1}}
\newcommand{\bkNN}{\mat{k}}
\newcommand{\T}{\ensuremath{\textit{T}}}
\newcommand{\Dir}{\ensuremath{\text{Dir}}}

\newcommand{\Bel}[0]{\text{Bel}}
\newcommand{\on}[0]{\text{on}}
\newcommand{\off}[0]{\text{off}}
\newcommand{\ith}[0]{\text{$i^\text{th}\;$}}
\newcommand{\jth}[0]{\text{$j^\text{th}\;$}}
\newcommand{\kth}[0]{\text{$k^\text{th}\;$}}
\newcommand{\wth}[0]{\text{$w^\text{th}\;$}}
\newcommand{\dth}[0]{\text{$d^\text{th}\;$}}
\newcommand{\nth}[0]{\text{$n^\text{th}\;$}}
\newcommand{\tth}[0]{\text{$t^\text{th}\;$}}

\newcommand{\half}[0]{\text{\tiny$\frac{1}{2}$}}
\newcommand{\onethird}[0]{\text{\tiny$\frac{1}{3}$}}
\newcommand{\twothirds}[0]{\text{\tiny$\frac{2}{3}$}}

\newcommand{\real}{\mathbb{R}}
\newcommand{\reals}{\mathbb{R}}
\newcommand{\Expectation}{\mathbb{E}}
\newcommand{\expectation}{\mathbb{E}}
\newcommand{\obs}{\mathrm{obs}}
\newcommand{\tinyhalf}{\mbox{\tiny $\frac{1}{2}$}}

\newcommand{\tick}{\ding{51}}
\newcommand{\btick}{\ding{52}}
\newcommand{\cross}{\ding{55}}
\newcommand{\bcross}{\ding{56}}
\newcommand{\given}{\,|\,}

\renewcommand{\vec}[1]{\mathbf{#1}}
%\newcommand{\balpha}{\mbox{\boldmath $\alpha$}}
%\newcommand{\btheta}{\mbox{\boldmath $\theta$}}
%\newcommand{\bphi}{\mbox{\boldmath $\phi$}}
%\newcommand{\bn}{\mbox{\boldmath $n$}}
\newcommand{\DCM}{\mbox{DCM}}

  
% I keep MarcusStyle.sty in $HOME/texmf/tex/latex, because 
% kpsewhich -var-value=TEXMFHOME says $HOME/texmf is where latex looks.

%\setlength{\parskip}{2mm plus 1mm minus 1mm}
\renewcommand{\arraystretch}{1.5}
\renewcommand*{\familydefault}{\sfdefault}

\title{question about likelihood of an image made by generating two regions separately and putting them together}
\author{marcus}
\date{}
\begin{document}
\maketitle

\section{setup}
Consider a vector whose elements are categorical variables $x \in \mathcal{X} $ where $\mathcal{X} = \{X_1,\ldots,X_K \}$
\begin{align*}
\bx &= (x_1,x_2, \ldots, x_N)
\intertext{We will use $\mathbb{X}$ to denote the same as a set (ie. without the ordering imposed by the vector):}
\mathbb{X} &= x_1,x_2, \ldots, x_N
\intertext{Denote the number of times  each category occurs (its "count") by}
\bn &= n_1,\ldots,n_K  
\intertext{and the total count}
N &= \sum_k^K n_k
\end{align*}



\subsection{split}
Now consider splitting the ``image'' $\bx$ into two parts, e.g. are some pixel mid-way along its length:
\begin{align*}
\bx =& (\by , \bz)
\intertext{To connect this to radioblobs, think of $\by$ as the values inside a putative blob region and $\bz$ as the rest (the "background"). 
At any rate, I initially thought one could just write
}
P(\bx) =& P(\by , \bz)\\
=& P(\by) \; P(\bz)
\end{align*}

But it seems that can't be right: imagine the ``image'' is binary
($\mathcal{X}=\{0,1\}$), our split point is just the halfway
mark. Suppose the left hand ($\by$) has rather more zeros by chance,
and the right ($\bz$) has more ones.

The worse this skew is, the more $P(\by, \bz)$ goes down since both
halves are getting less likely, but $P(\bx)$ stays the same, so they
can't be equal!

This seems to true for any distribution - it's not some kind of
Dirichlet weirdness.

{\color{red}{\bf This is just {\em stupid}:  what is my idiotic error?!}}


The vector \[\mathtt{000000011111111111111}\] has exactly the same
probability using a fair coin as \[\mathtt{10110111101111011100}\]
but what about the probability of 
tossing \[\mathtt{0000000} \;\textit{ followed by  } \;\mathtt{11111111111111}\]

My gut says $P(\bx) \neq P(\by , \bz)$ because the second is the
probability not just of the values but the values {\it split into those two
groups}. But that's just fluffy still.

Feel I'm mixing up SETS, VECTORS, and COUNTS, somehow.............

\section{Fun with Dirichlet}
Dirichlet Compound Multinomial (DCM) distribution has parameters
$\balpha = \alpha_1,..,\alpha_K$ with $\alpha_k > 0$ and we denote $ A
= \sum_k^K \alpha_k$.

Under the DCM distribution, we have that
\begin{align}
P(\mathbb{X} |\balpha) &= \frac{\Gamma(A)}{\Gamma(N+A)} \prod_k \frac{\Gamma(n_k+\alpha_k)}{\Gamma(\alpha_k)}  \label{eq:DCMvalues}
\end{align}

The gamma function just generalises the factorial to the reals: at
positive integer values they match except that the argument for gamma is
larger by one: $\Gamma(z) = (z-1)!$.

The probability of {\it counts} is almost the same, but with a multiplier:
\begin{align}
P(\bn |\alpha) &= \frac{N!}{\prod_k (n_k!)} \;
 \frac{\Gamma(A)}{\Gamma(N+A)} \;\prod_k \frac{\Gamma(n_k+\alpha_k)}{\Gamma(\alpha_k)}  \label{eq:DCMcounts}
\end{align}
I think I'm right in saying that the multiplier is the ``multinomial coefficient''
\[
\frac{N!}{\prod_k (n_k!)} \;
\;\;\; = \;\;\; 
{n \choose n_1, n_2,\ldots,n_K}
\]
which is a generalisation of the more familiar binomial coefficient "${n \choose k}$".

\subsection{a more succinct notation?}

The multinomial coefficient can be rewritten in terms of gamma functions:
\[
\frac{N!}{\prod_k (n_k!)} \;
\;\;\; = \;\;\; 
\frac{\Gamma(N+1)}{\prod_k \Gamma(n_k+1)}
\]
which has the same form as the other terms now.

Perhaps we can simplify things then. Let's define a function
\[
 \Omega(\bc) \;\;= \;\;\frac{\Gamma(C)}{\prod_k \Gamma(c_k)} \;\;=\;\; \frac{(C-1)!}{\prod_k (c_k-1)! }
\]
where $C=\sum_k c_k$.
Note the effect of incrementing one of the counts, say the one indexed $k^\star$:
\begin{align*}
\Omega(\bc) &\longrightarrow \Omega(\bc) \times \frac{C}{c_{k^\star}}
\intertext{whereas {\em de}crementing would change it to}
&\longrightarrow  \Omega(\bc) \times \frac{c_{k^\star}-1}{C-1}
\end{align*}


I can now write the DCM probability in equation \ref{eq:DCMcounts} using $\Omega$. Terms that come from the multinomial coefficient are shown in {\color{blue}blue}.
\begin{align}
P(\bn |\balpha) &= 
\frac{{\color{blue}\Omega(\bn+1)} \;\; \Omega(\balpha)}{ \Omega(\bn + \balpha)}
\label{eq:DCMcounts_inOmega}
\end{align}

Incrementing the $k^{\star}$-th count changes this to:
\begin{align}
P(\bn \mid \balpha) &\longrightarrow  
P(\bn \mid \balpha)\;\; \times \;\;{\color{blue}\bigg( \frac{N+1}{n_{k^\star}+1} 
\bigg)}
\;\,\bigg(\frac{n_{k^\star}+\alpha_{k^\star}}{N+A}
\bigg)\label{eq:increment}
\intertext{whereas {\em de}crementing would change it to}
&\longrightarrow  
P(\bn \mid \balpha)\;\; \times \;\;{\color{blue}\bigg( \frac{n_{k^\star}}{N} 
\bigg)}
\;\,\bigg(\frac{N+A-1}{n_{k^\star}+\alpha_{k^\star}-1} 
\bigg)\label{eq:decrement}
\end{align}


\subsection{moving the boundary}
Consider the ``split'' image again, with the left-hand side denoted
$\by$ and the right by $\bz$.

The effect of ``moving the boundary'' to the right by one pixel is to
take a single value that was in $\bz$ and move it into $\by$
instead. For generality, I'm going to distinguish between the
$\balpha$ used in generating $\by$ and the one used for $\bz$ too.

Using equations \ref{eq:increment} and \ref{eq:decrement}, the log likelihood of $\by$ changes by
\begin{align*}
\log P(\bn_\Delta^y \mid \balpha^y) -\log P(\bn^y \mid \balpha^y) 
&=
{\color{blue}\log(N^y+1) - \log(n^y_{k^\star}+1)}
+ \log(n^y_{k^\star}+\alpha^y_{k^\star}) - \log(N^y+A^y)
\intertext{whereas that of $\bz$ changes by}
\log P(\bn_\Delta^z \mid \balpha^z) - \log P(\bn^z \mid \balpha^z) 
&=
{\color{blue}\log(n^z_{k^\star}) - \log(N^z)} + \log(N^z+A^z-1) - \log(n^z_{k^\star}+\alpha^z_{k^\star}-1)
\end{align*}

SO IF (big if...) $P(\by,\bz \mid \text{model}) = P(\by \mid
\balpha^y) \, P(\bz \mid \balpha^z) $, this means the overall log
likelihood changes by
\begin{align*}
\Delta \log P &= \log P(\bn_\Delta^y,\bn_\Delta^z) - \log P(\bn^y,\bn^z) \\
&= {\color{blue}  
  \log \bigg(\; 
  \frac{N^y+1}{n^y_{k^\star}+1} \;\;
  \frac{n^z_{k^\star}}{N^z} \bigg)
  } 
\; + \; 
{\color{green}  
\log \bigg(\frac{n^y_{k^\star}+\alpha^y_{k^\star}}{N^y+A^y} \bigg) 
}
\; + \;
\log \bigg(\frac{N^z+A^z-1}{n^z_{k^\star}+\alpha^z_{k^\star}-1}
\bigg)
\end{align*}



\subsection{Bayes factor}
Alternatively...

Here's the Bayes factor comparing the likelihood of $\by$ under the two possible $\balpha$ priors:
\begin{align*}
\frac{P(\bn^y \mid \balpha^y)}{P(\bn^y \mid \balpha^z)} \;\;
 &= \;\;
\frac{{\color{blue}\Omega(\bn^y+1)} \;\; \Omega(\balpha^y)}{ \Omega(\bn^y + \balpha^y)} \;\;\;
\frac{ \Omega(\bn^y + \balpha^z)}{{\color{blue}\Omega(\bn^y+1)} \;\; \Omega(\balpha^z)} 
\intertext{Note that the effect of the multinomial coefficient {\it cancels out}, leaving}
&= \;\;\frac{\Omega(\balpha^y)}{\Omega(\balpha^z)} 
\;\;\;
\frac{ \Omega(\bn^y + \balpha^z)}{ \Omega(\bn^y + \balpha^y)} 
\end{align*}

We (if this is correct) have been calculating the log of this quantity.


So let's consider the effect of moving the boundary on the Bayes factor:
\begin{align*}
\frac{P(\bn_\Delta^y \mid \balpha^y)}{P(\bn_\Delta^y \mid \balpha^z)}
\;\; &= \;\; \frac{\Omega(\balpha^y)}{\Omega(\balpha^z)} 
\;\;\;
\frac{ 
 \Omega(\bn^y + \balpha^z) \;\times\; \frac{N^y+A^z}{n^y_{k^\star}+\alpha^z_{k^\star}}
}{
 \Omega(\bn^y + \balpha^y) \;\times\; \frac{N^y+A^y}{n^y_{k^\star}+\alpha^y_{k^\star}}
} 
\intertext{where $\bn_\Delta$ denotes the {\it new} counts. So...}
\text{new BF} \;\;&= \;\;\text{old BF} \; \times \; 
\bigg(
 \frac{N^y+A^z}{n^y_{k^\star}+\alpha^z_{k^\star}}
\bigg)
\; \bigg(\frac{n^y_{k^\star}+\alpha^y_{k^\star}}{N^y+A^y}
\bigg)
\intertext{so in log space, here's how the Bayes Factor changes when a single pixel of class $X_{k^\star}$ gets moved from $\bz$ into $\by$:}
\Delta \text{BF} \;\;&= \;\; 
\log\bigg( \frac{N^y+A^z}{n^y_{k^\star}+\alpha^z_{k^\star}}
\bigg) 
\; + \; 
{\color{green}  
\log\bigg(\frac{n^y_{k^\star}+\alpha^y_{k^\star}}{N^y+A^y}
 \bigg)
}
\end{align*}

{\color{red} RED THOUGHT: how about we define a source region as simply some set of
connected pixels, and individually ask all the pixels at the
borderline (both out and in) whether they want to join / leave?

What would happen?...}



\end{document}
